\documentclass{article}
\usepackage{multicol}
\usepackage{xcolor}
\usepackage{graphicx}
\linespread{1.35}
\usepackage{amsmath}
\usepackage{color}
\usepackage{tikz}
\usetikzlibrary{arrows,automata}

\begin{document}

\begin{flushright}
 \texttt{Finite Automata} \hspace*{0.1cm}\textbf{$|$} \hspace*{0.1cm} \textbf{105}\hspace*{0.1cm}
\end{flushright}
\vspace*{0.5cm}

\begin{center}
\begin{tabular}{ccccc}
 \hline

 \hline

 \hline

 \hline
 &  \multicolumn{2}{c}{$I/P = 0$ } &  \multicolumn{2}{c}{$I/P = 1$}  \\
  \cline{2-3}                         \cline{4-5}
 $Present State$ &   $Next State$  & $O/P$ &  $Next State$  & $O/P$\\
\hline
$q_20$             &  $q_11$  &  $1$  &  $q_21$  & $1$ \\
$q_21$             &  $q_11$  &  $1$  &  $q_21$  & $1$ \\
$q_3$             &  $q_20$  &  $1$  &  $q_0$  & $1$ \\

 \hline

 \hline

 \hline

 \hline
\end{tabular}
\end{center}

\vspace*{0.1cm}
The converted Moore machine is\\
\vspace*{0.1cm}

\begin{center}
\begin{tabular}{cccc}
 \hline

 \hline

 \hline

 \hline
 & \multicolumn{2}{c}{$Next State$}\\
 \cline{2-3}
 $State$ &  $I/P=0$ & $I/P=1$  &  $O/P$\\
\hline
$\rightarrow q_0$  &    $q_0$   &  $q_10$   &  $1$ \\
$q_10$              &    $q_3$   &  $q_3$   &  $0$ \\
$q_11$              &    $q_3$   &  $q_3$   &  $1$ \\
$q_20$              &    $q_11$   &  $q_21$   &  $0$ \\
$q_21$              &    $q_11$   &  $q_21$   &  $1$ \\
$q_3$              &    $q_20$   &  $q_0$   &  $1$ \\
 \hline

 \hline

 \hline

 \hline
\end{tabular}
\end{center}

\vspace*{0.3cm}


To get rid of the problem of occurrence of a null string, we need to include another state, $q_a$, with
the same transactions as that of q0 but with output 0.\\
\hspace*{0.5cm} The modified final Moore machine equivalent to the given Mealy machine is\\

\begin{center}
\begin{tabular}{cccc}
 \hline

 \hline

 \hline

 \hline
 & \multicolumn{2}{c}{$Next State$}\\
 \cline{2-3}
 $State$ &  $I/P=0$ & $I/P=1$  &  $O/P$\\
\hline
$\rightarrow q_a$ & $q_0$  & $q_10$  & 0\\
$q_0$ & $q_0$ & $q_10$ &1\\
$q_10$ & $q_3$ & $q_3$ &0\\
$q_11$ & $q_3$ & $q_3$ &1\\
$q_20$ & $q_11$ & $q_21$ &0\\
$q_21$ & $q_11$ & $q_21$ &1\\
$q_3$  & $q_20$ & $q_0$ &1\\
 \hline

 \hline

 \hline

 \hline
\end{tabular}
\end{center}

\vspace*{0.3cm}

17. Convert the following Mealy Machine to a Moore Machine.  \hspace*{0.5cm} [WBUT 2008]\\

\vspace*{0.3cm}
\begin{center}
\begin{tabular}{ccccc}
 \hline

 \hline

 \hline

 \hline
 &  \multicolumn{2}{c}{ Next State   $I/P = 0$ } &  \multicolumn{2}{c}{ Next State   $ I/P = 1$}  \\
  \cline{2-3}                         \cline{4-5}
 $Present State$ &State & Output &  State & Output\\
\hline
$Q_1$   &  $q_2$  &  $1$  &  $q_1$  & $0$ \\
$Q_2$   &  $q_3$  &  $0$  &  $q_4$  & $1$ \\
$Q_3$   &  $q_1$  &  $0$  &  $q_4$  & $0$ \\
$Q_4$   &  $q_3$  &  $1$  &  $q_2$  & $1$ \\
 \hline

 \hline

 \hline

 \hline
\end{tabular}
\end{center}


\newpage

\begin{flushleft}
    \textbf{106}\hspace*{0.1cm} \textbf{$|$} \hspace*{0.1cm} \texttt{Introduction to Automata Theory, Formal Languages and Computation}
  \end{flushleft}
  \vspace*{0.5cm}

\textbf{Solution:} $Q_3$ and $_4$ as next states produce outputs 0 and 1, and so the states are divided into $Q_30$, $Q_31$
and $Q_40$, $Q_41$. Thus, the constructing Moore machine contains six states. The Moore machine becomes\\

\begin{center}
\begin{tabular}{cccc}
 \hline

 \hline

 \hline

 \hline
 & \multicolumn{2}{c}{$Next State$}\\
 \cline{2-3}
 $State$ &  $I/P=0$ & $I/P=1$  &  $Output$\\
\hline
$Q_1$  &$Q_2$  & $Q_1$  &  0\\
$Q_2$  &$Q_30$ & $Q_41$& 1\\
$Q_30$ &$Q_1$  & $Q_40$& 0\\
$Q_31$ &$Q_1$  & $Q_40$& 1\\
$Q_40$ &$Q_31$ & $Q_2$& 0\\
$Q_41$ &$Q_31$ & $Q_2$& 1\\
 \hline

 \hline

 \hline

 \hline
\end{tabular}
\end{center}

\vspace*{0.3cm}

18. From the following Mealy machine, find the equivalent Moore machine. Check whether the Mealy
machine is a minimal one or not. Give proper justification to your answer. \hspace*{7cm} [WBUT 2007]\\

\begin{center}
\begin{tabular}{ccccc}
 \hline

 \hline

 \hline

 \hline
 &  \multicolumn{2}{c}{$I/P = 0$ } &  \multicolumn{2}{c}{$I/P = 1$}  \\
  \cline{2-3}                         \cline{4-5}
 $Present State$ &   $Next State$  & $O/P$ &  $Next State$  & $O/P$\\
\hline
$S_1$ & $S_2$ & 0 & $S_1$ &0\\
$S_2$ & $S_2$ & 0 & $S_3$ &0\\
$S_3$ & $S_4$ & 0 & $S_1$ &0\\
$S_4$ & $S_2$ & 0 & $S_5$ &0\\
$S_5$ & $S_2$ & 0 & $S_1$ &1\\

 \hline

 \hline

 \hline

 \hline
\end{tabular}
\end{center}

\vspace*{0.3cm}

\emph{
\textbf{Solution:}\\
}

i) In the Mealy machine, $S_1$ as the next state produces output 0 for some cases and produces
output 1 for one case. For this reason, the state $S_1$ is divided into two parts: $S_10$ and $S_11$. All the
other states produce output 0.\\
To get rid of the problem of occurrence of a null string, we need to include another state, $S_a$,
with the same transactions as that of $S_10$ but with output 0.\\
\hspace*{0.5cm} The modified final Moore machine equivalent to the given Mealy machine will be as follows.\\

\hspace*{0.5cm}The converted Moore machine is\\

\vspace*{0.4cm}
\begin{center}
\begin{tabular}{cccc}
 \hline

 \hline

 \hline

 \hline
 & \multicolumn{2}{c}{$Next State$}\\
 \cline{2-3}
 $State$ &  $I/P=0$ & $I/P=1$  &  $Output$\\
\hline
$S_a$  & $S_2$   & $S_10$ & 0\\
$S_10$ & $S_2$  & $S_10$ & 0\\
$S_11$ & $S_2$  & $S_10$& 1\\
$S_2$  & $S_2$   & $S_3$  & 0\\
$S_3$  & $S_4$   & $S_10$ & 0\\
$S_4$  & $S_2$   & $S_5$  & 0\\
$S_5$  & $S_2$   & $S_11$ & 0\\
 \hline

 \hline

 \hline

 \hline
\end{tabular}
\end{center}


\newpage

\begin{flushright}
 \texttt{Finite Automata} \hspace*{0.1cm}\textbf{$|$} \hspace*{0.1cm} \textbf{107}\hspace*{0.1cm}
\end{flushright}
\vspace*{0.5cm}

ii) All the states are 0 equivalents.\\

\hspace*{4cm} $P_0 = \{S_1S_2S_3S_4S_5\}$ \\

\hspace*{0.5cm} For string length 1, all the states produce output 0 except $S_5$.\\

\hspace*{4cm} $P_1 = \{S_1S_2S_3S_4\}\{S_5\}$ \\

\hspace*{0.5cm} The next states of all the states (belong to the first subset) for all inputs belong to one set
except $S_4$. The modified partition is\\

\hspace*{4cm} $P_2 = \{S_1S_2S_3\}\{S4\}\{S5\}$ \\

\hspace*{0.5cm} By this process, $P_3 = \{S1S2\}\{S3\}\{S4\}\{S5\}$ \\

\hspace*{4cm} $P_4 = \{S_1\}\{S_2\}\{S_3\}\{S_4\}\{S_5\}$ \\

\hspace*{0.5cm} The machine is a reduced machine as the number of subsets of the machine is the same
as the number of states of the original Mealy machine. Hence, the machine is a minimal
machine.\\

\vspace*{0.4cm}

19. Convert the following Moore machine into an equivalent Mealy machine by the transitional
format.\\

\vspace*{0.1cm}
\begin{center}
\section{picture}
\includegraphics[width=6cm,height=3cm]{107.png}
\end{center}

\textbf{Solution:}\\
i) In this machine, A is the beginning state. So start from A. For A, there are three incoming
arcs, from A to A with input b, one in the form of start-state indication with no input, and the
last is from D to A with input a. State A is labelled with output 1. As the start-state indication
contains no input, it is useless and, therefore, keep it as it is.\\

\hspace*{0.5cm} Modify the label of the incoming edge from D to A and from A to A including the output
of state A. So, the label of the incoming state will be D to A with label a/1 and A to A with
label b/1.\\

\vspace*{0.2cm}
ii) State B is labelled with output 0. The incoming edges to the state B are from A to B with input
a and from D to B with input b.\\
\hspace*{0.5cm} Modify the labels of the incoming edges including the output of state B. So, the labels of
the incoming states will be A to B with label a/0 and from D to B with label b/0.\\

\vspace*{0.2cm}
iii) State C is labelled with output 0. There are two incoming edges to this state, from B to C with
input b and from C to C with input a.\\
\hspace*{0.5cm} The modified label will be B to C with label b/0 and C to C with label a/0.\\


\newpage

\begin{flushleft}
    \textbf{108}\hspace*{0.1cm} \textbf{$|$} \hspace*{0.1cm} \texttt{Introduction to Automata Theory, Formal Languages and Computation}
  \end{flushleft}
  \vspace*{0.5cm}

iv) State D is labelled with output 1. There are two incoming edges to this state, from B to D with
input a and from C to D with input b.\\
\hspace*{0.5cm} The modified label will be B to D with label a/1, and C to D with label b/1.\\
\hspace*{0.5cm} The converted Mealy machine will be\\

\vspace*{0.3cm}
\begin{center}
\section{picture}
\includegraphics[width=6cm,height=3cm]{108-1.png}
\end{center}

20. Convert the following Mealy machine into an equivalent Moore machine by the transitional
format.\\

\vspace*{0.3cm}
\begin{center}
\section{picture}
\includegraphics[width=6cm,height=3cm]{108-2.png}
\end{center}

\textbf{Solution:} The machine contains four states. Let us start from the state A. The incoming
edges to this state are from D to A with label a/0. There is no difference in the outputs of the
incoming edges to this state, and so in the constructing Moore machine the output for this
state will be 0.\\

\vspace*{0.1cm}
\begin{center}
\section{picture}
\includegraphics[width=6cm,height=3cm]{108-3.png}
\end{center}


For the state B, the incoming edges are B to B with label a/1, from A to B with label a/0, and from
C to B with label b/1.\\
\hspace*{0.5cm} We get two different outputs for two incoming edges (B to B output 1, A to B output 0). So, the
state B will be divided into two, namely, B0 and B1. The outgoing edges are duplicated for both
the states generated from B. The modified machine is\\

\end{document} 